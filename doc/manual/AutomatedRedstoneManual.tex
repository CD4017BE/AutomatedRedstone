% !TEX TS-program = pdflatex
% !TEX encoding = UTF-8 Unicode

% This is a simple template for a LaTeX document using the "article" class.
% See "book", "report", "letter" for other types of document.

\documentclass[11pt]{article} % use larger type; default would be 10pt

\usepackage[utf8]{inputenc} % set input encoding (not needed with XeLaTeX)

%%% Examples of Article customizations
% These packages are optional, depending whether you want the features they provide.
% See the LaTeX Companion or other references for full information.

%%% PAGE DIMENSIONS
\usepackage{geometry} % to change the page dimensions
\geometry{a4paper} % or letterpaper (US) or a5paper or....
% \geometry{margin=2in} % for example, change the margins to 2 inches all round
% \geometry{landscape} % set up the page for landscape
%   read geometry.pdf for detailed page layout information

\usepackage{graphbox}
% \usepackage{graphicxbox}
\graphicspath{{img/}}
\usepackage{tabularx}
\usepackage{listings}
\usepackage{amsmath}

% \usepackage[parfill]{parskip} % Activate to begin paragraphs with an empty line rather than an indent

%%% PACKAGES
\usepackage{booktabs} % for much better looking tables
\usepackage{array} % for better arrays (eg matrices) in maths
\usepackage{paralist} % very flexible & customisable lists (eg. enumerate/itemize, etc.)
\usepackage{verbatim} % adds environment for commenting out blocks of text & for better verbatim
\usepackage{subfig} % make it possible to include more than one captioned figure/table in a single float
% These packages are all incorporated in the memoir class to one degree or another...

%%% HEADERS & FOOTERS
\usepackage{fancyhdr} % This should be set AFTER setting up the page geometry
\pagestyle{fancy} % options: empty , plain , fancy
\renewcommand{\headrulewidth}{0pt} % customise the layout...
\lhead{}\chead{}\rhead{}
\lfoot{}\cfoot{\thepage}\rfoot{}

%%% SECTION TITLE APPEARANCE
\usepackage{sectsty}
\allsectionsfont{\sffamily\mdseries\upshape} % (See the fntguide.pdf for font help)
% (This matches ConTeXt defaults)

%%% ToC (table of contents) APPEARANCE
\usepackage[nottoc,notlof,notlot]{tocbibind} % Put the bibliography in the ToC
\usepackage[titles,subfigure]{tocloft} % Alter the style of the Table of Contents
\renewcommand{\cftsecfont}{\rmfamily\mdseries\upshape}
\renewcommand{\cftsecpagefont}{\rmfamily\mdseries\upshape} % No bold!

%%% END Article customizations

\newcommand{\imgtex}{\begin{tabularx}{\textwidth}{@{}c@{ }X@{}}}

%%% The "real" document content comes below...

\title{Automated Redstone Manual}
\author{by CD4017BE}
\date{mod version: 3.0.1}

\begin{document}
 \maketitle
 \tableofcontents
 
 \section{About this manual}  
Ingame there already is basic information about the blocks and items available in their tooltips if you press SHIFT. So this manual contains the more detailed information about how things work in Automated Redstone as well as giving some examples. But crafting recipes are not available here, use the Just Enough Items (JEI) mod to view them ingame.

Most screenshots used here are edited to make them fit better into the document, so don't worry if your ingame should look a bit different. 

The information provided here refers to the mod version displayed above.

\newpage
\section{Signal transport}
This mod adds a few cables to the game that allow transport of (redstone-) signals: 
\subsection{Analog Redstone cables}
Analog redstone cables (currently named \bf 1-bit solid redstone wire\rm) will receive redstone signals from their input connections (blue) and transmit them further to their output connections (green). If it has multiple inputs it will choose the strongest signal for output. To control the signal transfer direction, these cables come in three variants:\\
\bf Input cables \rm will by default connect to neighboring (non replaceable) blocks in the world and receive signals from these. They will try to establish output connections to other cables if possible.\\
\bf Output cables \rm also connect to other blocks and will emit their signal to them. They will try to establish input connections to other cables.\\
\bf Transport cables \rm only connect to other cables and will establish their connections so that they transport signals from input to output cables. They also connect to some other devices in this mod on their own. If you already used the basic item/fluid transport pipes from \bf Inductive Automation \rm you will notice that the connection behavior is the same.\\
\includegraphics[width=\textwidth]{1_bit_pipes}\\
You can also sneak-right-click on cables to disconnect/reconnect them or cover them with solid opaque blocks by right-click (like in Inductive Automation).

The transmitted signal won't loose any strength and can be transported over theoretically unlimited distances. However there is a delay of one tick (50ms) every 16 blocks.

\subsection{8-bit Redstone}
8-bit redstone cables transmit 8 individual signal through one cable, where every signal can be either on or off (no strength in between). They only connect to other blocks that receive or emit  8-bit redstone and will orient their transfer direction similar to the analog transport cables. Disconnection and covering works the same like above.\\

\subsection{8-bit wireless transfer}
\imgtex
\includegraphics[align = t]{wireless_8_bit} & The 8-bit wireless connector comes in two variants: transmitter and receiver. By placing the block it will turn into a transmitter and a receiver block that is linked to it will be added to your inventory. If you break these blocks to place them somewhere else, they will stay linked to each other as long as you don't have both in item form at same time because they are just linked by knowing the position and dimension Id of the other block. If you manage to break the linkage, just craft them together into the original connector item again. Sneak-right-clicking a linked receiver / transmitter will remove both linked blocks simultanously and will also give you the original connector item back. (works like the interdimensional wormhole of InductiveAutomation)\\
\end{tabularx}
The \bf 8-bit wireless receiver \rm will emit the 8-bit singal that was received by its linked \bf 8-bit wireless transmitter \rm to connected devices. The linkage also works across dimensions, but only if both components are chunkloaded.

\section{Signal components}
\subsection{8-bit lever \& display}
An easy way to produce 8-bit signals is the \bf 8-bit lever \rm. It has a front side with 8 red switches and will output its state to all connected 8-bit components.\\
And 8-bit states can be displayed using a \bf 8-bit display \rm. It has three different display modes that can be changed by right-clicking the block's front: \\
In binary mode it will display the 8 states as individual lamps, arranged the same as the switches in the 8-bit lever are. In hexadecimal mode the state will be displayed as two digit hexadecimal number in range $00\dots FF$. And in decimal mode the state is displayed as positive three digit decimal number in range $000\dots255$\\
\includegraphics[width=\textwidth]{8_bit_io}\\

\subsection{Logic signal converter}
\imgtex
\includegraphics[align = t]{logic_op} & The \bf logic signal converter \rm is used to combine or split multiple signals in digital from. In its gui you can set which sides of the block should be input or output and set the filter value that should be applied on each signal as AND-operation.\\
\end{tabularx}

The internal logic of the block does the following:\\
At first all signals received at its input sides will be filtered. If it receives a 8-bit signal, it will be combined with the specified value using a logical AND-operation. Otherwise if it receives a normal redstone signal with a $stength > 0$ then the resulting 8-bit state will be equal to the specified filter value or 0 if $strength = 0$.\\
Then all filtered input states are combined with logical OR-operations. \\
Finally all output sides will emit the logical AND-operation between the resulting state and the individual filter value as a 8-bit signal and it will additionally emit a normal redstone signal of strength 15 if that 8-bit signal is not zero.

\subsection{Arithmetic signal converter}
\imgtex
\includegraphics[align = t]{arithmetic_op} & The \bf arithmetic signal converter \rm is used to combine or split multiple signals in analog form. Here you can also set which sides should be input or output and set a multiplier that should be applied on each signal, as well as an offset value.\\
\end{tabularx}

The internal logic of the block does the following:\\
At first all signals received at its input sided are converted into numbers and then multiplied with the specified multiplicator. For 8-bit signals it will use the represented binary number and for normal redstone signals it will use its $strength * 16$ (result: $0\dots240$).
The multiplication results and the specified offset value are added together. And finally all output sides will emit that sum divided by their specified multiplier with the result clipped to a range of $0\dots255$. As 8-bit value that is just the binary representation of the number and as normal redstone value the strength will be the value divided by 16 rounded down.

\section{Inventory control}
\subsection{Inventory connectors}
\subsection{Inventory reader}
\subsection{Item translocator}
\it Documentation coming soon!\rm

\section{Integerated redstone circuits}
\imgtex
\includegraphics[align = t]{circuit} & The \bf Redstone Circuit \rm block allows you to put complicated redstone logic circuits into one single block, making them very compact and a bit cheaper in redstone cost. To define what the block should do you need to program it using the \bf Circuit Programmer \rm (explained later).\\
\end{tabularx}

\subsection{Creating \& using a circuit}
\imgtex
\includegraphics[align = t]{assembler} & Also your circuit needs some logic gates, IO-ports, and counter to perform the program. These are installed using the \bf Circuit Assembler\rm .
The exact amount of these parts depend on your program and is displayed in the Circuit Programmer when trying to put the program on the circuit.\\
\end{tabularx}

For creating circuits in the assembler put the citcuit item into the top right slot. Adding logic gates requires 1 redstone dust each, adding counters requires 2 netherquartz each and adding available signal outputs requires 1 lever each. Put the ingredients into the 3 left slots and click on the big button int the middle to insert them. Circuits can also be disassembled again on the left side of the assembler's GUI giving you the parts back. The circuit's internal logic can contain up to 128 gates and/or up to 8 8-bit counter and up to 16 outputs.\\
\includegraphics[width = \textwidth]{assembler_gui}\\
\it About the colored tab on the very left read section 2.1 "Machine Configuration" in the Inductive Automation Manual. \rm\\

When placed the circuit block has 16 1-bit channels for each input and output, so to use all of them with only 6 available block faces, you would need to use 8-bit wire or place multiple circuits next to each other.
Therefore they are bundled into four 8-bit chunks (2 in + 2 out). 

In the circuits's GUI you can set for each block face, which 8-bit chunk to use and a HEX-AND-gate filter that should be applied (like in \bf logic signal converter\rm ). So applying a normal redstone signal will set all bits contained in the filter to on and for output a redstone signal will be emitted if at least one of the bits contained in the filter is on.

The circuit's internal logic does the following:
The state of all gates is updated each operation tick, based on the individually defined input states and the type of logic operation they represent. This is done in the order they appear in the programm using the input states present at that time, so referring to previous gates will use their just updated state and referring to later gates (or themself) will use the state they had after last update tick.

After that the counters will count up by one if the gate defined as SET-input is on, otherwise if the gate defined as RESET-input is on they will go back to 0. They also go back to 0 if they reach above 255. Their current 8-bit counting states are stored in gate slots $64\dots127$ using 8 for each counter, so when using counters these slots are occupied and can't be used for logic gates.

Finally the circuit's outputs will be set to the state of the individual gates they are bound to.

The operation tick interval can be set in the circuit's GUI. It also contains a ON/OFF button to run or pause updating and a RESET button to set all gates to off.

\subsection{Programming}
\it not finished yet, just copy pasted from curse!\rm\\
Logic Gates:
The logic gates are defined on the left side of the circuit programmer GUI.
Every line defines one of the following logic gates:
OR-gate ($+$): output on if any given input on.
NOR-gate ($-$): output off if any given input on.
AND-gate ($\&$): output on if all given inputs on.
NAND-gate ($*$): output off if all given inputs on.
XOR-gate ($/$): output on if an uneven number of given inputs on.
XNOR-gate ($\backslash$): output on if an even number of given inputs on.
Input-port ($\% $): output on if the given input channel of the circuit receives a redstone signal.
$x = k (\# = )$: output on if the given byte equals the given number.
$x < k (\# < )$: output on if the given byte is smaller than the given number.
$x > k (\# > )$: output on if the given byte is greater than the given number.
The line starts with the logic gate symbol (shown in brackets) and is followed by zero or more parameters which are seperated by colons (,).
For the OR,NOR,AND,NAND,XOR and XNOR command every parameter is a number in range 0-127 that defines which bit of the circuit's logic RAM to use as input for this gate.
For logic gates the bit ID is the same as the line number in the program and for counters the bit ID is 64 + (counter number * 8) + counter bit.
The Input command requires one number in range 0-15 as parameter that defines which of the 16 input channels to use.
The comparator commands require a number in range 0-15 as parameter before the comparator sign that defines which 8-bit-block of the circuit's logic RAM to use, and a number in range 0-255 as parameter after the comparator sign that defines the number to compare that 8-bit-block with.
Counter:
Every counter will increase its value by 1 every circuit tick if its set-bit is on.
If the value of a counter reaches 256 or if its reset-bit is on it will be set back to 0.
The set and reset bits of all the 8 counters can be set in the first two columns on the right of the GUI.
Set them to -1 if a counter is not used.
Output:
The numbers in the third and fourth column on the right of the GUI define which bit of the circuit's logic RAM to use for each of the 16 output channels.
Set it to -1 if a output channel is not used.


\subsection{Example programms}
\it comming soon!\rm

\end{document}